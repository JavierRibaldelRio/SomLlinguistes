% Options for packages loaded elsewhere
\PassOptionsToPackage{unicode}{hyperref}
\PassOptionsToPackage{hyphens}{url}
%
\documentclass[
]{article}
\usepackage{lmodern}
\usepackage{amssymb,amsmath}
\usepackage{ifxetex,ifluatex}
\ifnum 0\ifxetex 1\fi\ifluatex 1\fi=0 % if pdftex
  \usepackage[T1]{fontenc}
  \usepackage[utf8]{inputenc}
  \usepackage{textcomp} % provide euro and other symbols
\else % if luatex or xetex
  \usepackage{unicode-math}
  \defaultfontfeatures{Scale=MatchLowercase}
  \defaultfontfeatures[\rmfamily]{Ligatures=TeX,Scale=1}
\fi
% Use upquote if available, for straight quotes in verbatim environments
\IfFileExists{upquote.sty}{\usepackage{upquote}}{}
\IfFileExists{microtype.sty}{% use microtype if available
  \usepackage[]{microtype}
  \UseMicrotypeSet[protrusion]{basicmath} % disable protrusion for tt fonts
}{}
\makeatletter
\@ifundefined{KOMAClassName}{% if non-KOMA class
  \IfFileExists{parskip.sty}{%
    \usepackage{parskip}
  }{% else
    \setlength{\parindent}{0pt}
    \setlength{\parskip}{6pt plus 2pt minus 1pt}}
}{% if KOMA class
  \KOMAoptions{parskip=half}}
\makeatother
\usepackage{xcolor}
\IfFileExists{xurl.sty}{\usepackage{xurl}}{} % add URL line breaks if available
\IfFileExists{bookmark.sty}{\usepackage{bookmark}}{\usepackage{hyperref}}
\hypersetup{
  pdftitle={Notícia},
  pdfauthor={Javier Ribal del Río},
  hidelinks,
  pdfcreator={LaTeX via pandoc}}
\urlstyle{same} % disable monospaced font for URLs
\usepackage[margin=1in]{geometry}
\usepackage{graphicx,grffile}
\makeatletter
\def\maxwidth{\ifdim\Gin@nat@width>\linewidth\linewidth\else\Gin@nat@width\fi}
\def\maxheight{\ifdim\Gin@nat@height>\textheight\textheight\else\Gin@nat@height\fi}
\makeatother
% Scale images if necessary, so that they will not overflow the page
% margins by default, and it is still possible to overwrite the defaults
% using explicit options in \includegraphics[width, height, ...]{}
\setkeys{Gin}{width=\maxwidth,height=\maxheight,keepaspectratio}
% Set default figure placement to htbp
\makeatletter
\def\fps@figure{htbp}
\makeatother
\setlength{\emergencystretch}{3em} % prevent overfull lines
\providecommand{\tightlist}{%
  \setlength{\itemsep}{0pt}\setlength{\parskip}{0pt}}
\setcounter{secnumdepth}{-\maxdimen} % remove section numbering

\title{Notícia}
\author{Javier Ribal del Río}
\date{2020-21}

\begin{document}
\maketitle

{
\setcounter{tocdepth}{4}
\tableofcontents
}
\hypertarget{definiciuxf3}{%
\section{Definició}\label{definiciuxf3}}

La notícia es un text \textbf{periodistic} és un text de tipus narratius
que té com objectiu informar de fets o accions d'interés generals
ocorreguts recientment.

\hypertarget{caracteruxedstiques}{%
\section{Característiques}\label{caracteruxedstiques}}

\begin{itemize}
\tightlist
\item
  Objetiva
\item
  Senzilla
\item
  Clara
\item
  Concisa
\item
  Llenguatge estàndar
\item
  Fonts fiables Per a transmetre credibilitat

  \begin{itemize}
  \tightlist
  \item
    Informació
  \item
    Documentació gràfica
  \item
    Escrita
  \item
    Sonora
  \end{itemize}
\end{itemize}

\hypertarget{parts}{%
\section{Parts}\label{parts}}

\hypertarget{autoria-o-redacciuxf3}{%
\subsection{Autoria o redacció}\label{autoria-o-redacciuxf3}}

Nom del periosdista o agència que a elaborat la notícia. Lloc de la
redacció.

\hypertarget{titular-o-tuxedtol}{%
\subsection{Titular o títol}\label{titular-o-tuxedtol}}

Enunciat breu, situat a l'encapçalament de manera molt destacada,
designa el contingut essencial d'una notícia.

\hypertarget{entradeta-o-lead}{%
\subsection{\texorpdfstring{Entradeta o
\emph{lead}}{Entradeta o lead}}\label{entradeta-o-lead}}

És el primer paràgraf, escrit amb una tipografia diferent, que conté les
dades essencials del fet o acció. Sereveix per a que el lector es puga
fer-se'n una idea completa sense llegir la resta del text.

\hypertarget{peu-de-foto}{%
\subsection{Peu de foto}\label{peu-de-foto}}

Comentari breu sobre la fotografia. Autoria de la foto

\hypertarget{cos}{%
\subsection{Cos}\label{cos}}

Part més ampla la cual presenta l' informació del fet o acció o
situació. Sol estar divit en paràgrafs, distribuïts en diferents
columnes

\hypertarget{informacio-del-cos}{%
\subsubsection{Informacio del cos}\label{informacio-del-cos}}

\begin{itemize}
\tightlist
\item
  Versemblant
\item
  Coherent
\item
  Cohesionada
\item
  Adequada
\end{itemize}

\hypertarget{caracteruxedstiques-de-la-informaciuxf3}{%
\section{Característiques de la
informació}\label{caracteruxedstiques-de-la-informaciuxf3}}

Totes les noticias han de contestar a les següents prenguntes:

\hypertarget{qui}{%
\subsection{Qui?}\label{qui}}

Protasgonista de la acció

\hypertarget{quuxe8}{%
\subsection{Què?}\label{quuxe8}}

El fet?

\hypertarget{on}{%
\subsection{On?}\label{on}}

El lloc

\hypertarget{quan}{%
\subsection{Quan?}\label{quan}}

El temps

\hypertarget{com}{%
\subsection{Com?}\label{com}}

Les circumstàncies

\hypertarget{per-quuxe8}{%
\subsection{Per què?}\label{per-quuxe8}}

La causa

\hypertarget{sis-w}{%
\subsection{Sis W}\label{sis-w}}

Aquest es el nom que rebrem el conjunt d'aquestes preguntes, en l'argot
del periodisme.

\begin{itemize}
\tightlist
\item
  Who?
\item
  What?
\item
  Where?
\item
  When?
\item
  How?
\item
  Why?
\end{itemize}

Això ve perquè totes les preguntes contenen la lletra ``w'' en angles.

La resposta a les quatre primeres preguntes(``Qui?'', ``Qué?'', ``On?''
i ``Quan?'') ha d'aparéixer en el titutlar o entradeta. Per contrapart
les dos ultimes (``Com?'' i ``Per què?'') poden aparéixer en el cos.

\hypertarget{estructura}{%
\section{Estructura}\label{estructura}}

Es diu que la estructura de la notícia te forma de piràmide invertida ja
que presenta m´es informació al principi de la notícia que cap al final
Això es fa per a \textbf{Dosificar la Informació}.

Volver a la Página Principal

\end{document}
